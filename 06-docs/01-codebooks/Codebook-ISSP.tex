\documentclass[12pt,english]{article}
\usepackage[usenames, dvipsnames]{xcolor}
\usepackage[top=2cm, bottom=2cm, left=2.5cm, right=2.5cm]{geometry}
\usepackage[T1]{fontenc}
\usepackage{inputenc}
\usepackage{parskip}
\setlength{\parindent}{0pt}
\usepackage{amsmath}
\usepackage{caption}
\usepackage{url}
\usepackage[bookmarksnumbered]{hyperref}
\usepackage{babel}
\usepackage{graphicx}
\usepackage{CormorantGaramond}
\usepackage{dcolumn}
\usepackage{setspace}
\onehalfspacing
\usepackage{titlesec}
\titleformat{\section}{\large\bfseries}{\thesection}{0.4em}{}
\titleformat{\subsection}{\normalfont\bfseries}{\thesubsection}{0.2em}{}
\usepackage{booktabs}
\setlength{\heavyrulewidth}{0.2em}
\usepackage{apacite}
\bibliographystyle{apacite}
\makeatletter
\renewcommand{\maketitle}{
  \begin{flushleft}
    {\huge\@title}\\
    \vspace{10pt}
    {\large\@author}\\
    {\@date}
    \vspace{40pt}
  \end{flushleft}
}
\makeatother
\usepackage{authblk}
\title{\textsc{Codebook ISSP Citizenship (round 2)}}
\author{Constantin Manuel Bosancianu}
\affil{WZB Berlin Social Science Center \\ \textit{Institutions and Political Inequality}}
\date{July 13, 2019}
\begin{document}
\maketitle

The data comes from round 2 of the \textit{Citizenship} module of the \textit{International Social Survey Programme}.\footnote{The vast majority of the merging code was written by Olga Leshchenko, for whose assistance I am grateful.} The data has been cleaned to the point where it is almost ready to use in statistical specifications. Only some of the variables have been used in the models presented in the course. The rest are made available to the participants for the purposes of exploration.

The following variables measured at the individual-level can be found in the data:

\begin{enumerate}
  \item \texttt{cnt}: 2-letter code for country;
  \item \texttt{country}: full name of country;
  \item \texttt{year}: year of data collection;
  \item \texttt{V41}: ``People like me don't have any say about what the government does'' (agree--disagree);
  \item \texttt{V42}: ``I don't think the government cares much what people like me think'' (agree--disagree);
  \item \texttt{V43}: ``I feel I have a pretty good understanding of the important political issues facing our country'' (agree--disagree);
  \item \texttt{V44}: ``I think most people in our country are better informed about politics and government than I am'' (agree--disagree);
  \item \texttt{poleff}: average of 4 items measuring political efficacy(\texttt{V41}--\texttt{V44})---higher values denote a greater sense of political efficacy;\footnote{Item \texttt{V43} has been reverse-coded so that it matches the other 3 items.}
  \item \texttt{female}: gender (1=woman; 0=man);
  \item \texttt{age10}: age of respondent, measured in decades (32 years = 3.2 decades);
  \item \texttt{married}: R. is married or in a civil partnership (1 = yes; 0 = no);
  \item \texttt{unmemb}: R. is currently a member of a trade union (1 = yes; 0 = no);
  \item \texttt{educ}: number of years of full-time education completed by R.;\footnote{All values larger than 30 were coded as missing.}
  \item \texttt{incquart}: the country-specific income quartile in which R. fits (1 = bottom 25\%; \dots; 4 = top 25\%);
  \item \texttt{degree}: highest completed educational level (0 = no formal education; 1 = primary school; 2 = lower secondary; 3 = upper secondary; 4 = post-secondary, non-tertiary; 5 = lower level tertiary, first stage; 6 = upper level tertiary (MA, PhD));
  \item \texttt{unemp}: R. is unemployed and looking for a job (1 = yes; 0 = no);
  \item \texttt{place}: R's self-placement on a social hierarchy scale (1 = Bottom 10\%; \dots; 10 = Top 10\%);
  \item \texttt{relatt}: R's attendance at religious services apart from special occasions (weddings, funerals) (1 = once a month or more; 2 = less than once a month; 3 = never);
  \item \texttt{voted}: voted in the previous election (1 = yes; 0 = no);
  \item \texttt{urban}: R. lives in a big city, or the suburbs of a big city (1 = yes; 0 = no).
\end{enumerate}

The data set also contains a host of country-level indicators:

\begin{enumerate}
  \item \texttt{gini10}: net Gini value, obtained from Frederick Solt's \textit{SWIID} data, version 8.1;\footnote{Value is expressed in 10-point units. Higher values denote more inequality.}
  \item \texttt{ti\_cpi}: Transparency International's \textit{Corruption Perceptions Index} (obtained from then January 2019 version of the \textit{Quality of Governance} data)---higher values denote less corruption;
  \item \texttt{unDens}: union density (obtained from Jelle Visser's \textit{ICTWSS} data, version 6.0, from June 2019);
  \item \texttt{pub\_finance}: extent to which public funding is available for campaigns for national office (higher values denote greater availability of public funding) (obtained from \texttt{V-DEM} data, version 9, original variable name is \texttt{v2elpubfin\_mean});
  \item \texttt{leg\_corrupt}: extent to which members of the legislature abuse their position for financial gain (higher values denote a greater degree of honesty) (obtained from \texttt{V-DEM} data, version 9, original variable name is \texttt{v2lgcrrpt\_mean});
  \item \texttt{pol\_power}: power is distributed in society according to socio-economic position (higher values denote a greater degree of equality) (obtained from \texttt{V-DEM} data, version 9, original variable name is \texttt{v2pepwrses});
  \item \texttt{fed}: federal political system (1 = either weak or strong federalism; 0 = unitary system) (obtained from \texttt{CPDS} data, August 2018 version); 
  \item \texttt{prop}: proportional electoral system (1 = PR; 0 = either FPTP or modified PR systems) (obtained from \texttt{CPDS} data, August 2018 version);
  \item \texttt{unemrate}: unemployment rate, percentage of civilian labor force (obtained from \texttt{CPDS} data, August 2018 version);
  \item \texttt{realgdpgr}: growth of real GDP, expressed as percent change from previous year (obtained from \texttt{CPDS} data, August 2018 version);
  \item \texttt{enep}: effective number of parties computed based on the Laakso and Taagepera (1979) formula (obtained from \texttt{CPDS} data, August 2018 version);
  \item \texttt{pres}: presidential regime (1 = presidential, semi-presidential dominated by president, or hybrid; 0 = parliamentary, semi-presidential dominated by parliament) (obtained from \texttt{CPDS} data, August 2018 version);
  \item \texttt{bic}: bicameral legislature, based on Lijphart's index of bicameralism (1 = medium-strong or strong bicameralism; 0 = unicameralism or weak bicameralism) (obtained from \texttt{CPDS} data, August 2018 version);
  \item \texttt{mdmh}: mean district magnitude for elections to the lower chamber of the legislature (obtained from the \textit{Database of Political Institutions}, February 2018 version);
  \item \texttt{welfGen}: welfare generosity index (aggregating unemployment benefits, sick pay, and pensions) (obtained from the \textit{Comparative Welfare Entitlements Dataset, December 2018 version});
  \item \texttt{gdp}: GDP per capita, PPP (expressed in current international USD) (obtained from \textit{World Bank} data);
  \item \texttt{wbgi\_gee}: \textit{World Bank}'s Government Effectiveness Index (from the ``Governance Indicators'' series) (obtained from then January 2019 version of the \textit{Quality of Governance} data);
  \item \texttt{polRile}: polarization on a Left--Right dimension, computed as the sum of the weighted deviations of parties from the ideological center, using party vote shares as weights. Computed based on \texttt{MARPOR} (manifesto) data, version 2018b;
  \item \texttt{polSocec}: polarization on a socio-economic dimension, computed in the same way as above. Obtained using \texttt{MARPOR} (manifesto) data, version 2018b;
  \item \texttt{lsq}: the Gallagher index of electoral disproportionality;\footnote{Obtained from the personal page of the author: \url{https://www.tcd.ie/Political_Science/people/michael_gallagher/ElSystems/Docts/ElectionIndices.pdf}.}
\end{enumerate}

\end{document}